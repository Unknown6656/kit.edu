\documentclass[pdftex,12pt,a4paper]{report}
\usepackage[utf8]{inputenc}
\usepackage{graphicx}
\usepackage[fleqn]{amsmath}
\usepackage[fleqn]{mathtools}
\usepackage{amssymb}
\usepackage{mathrsfs}
\usepackage{amsfonts}
\usepackage{geometry}
\usepackage{listings}
\usepackage{wrapfig}
\usepackage{multirow,array}
\usepackage{tabularx}
\usepackage{textcomp}
\usepackage[T1]{fontenc}
\usepackage{inconsolata}
\usepackage{color}
\usepackage{lipsum}
\usepackage{courier}
\usepackage{amsthm}
\usepackage{amsmath}
\usepackage{verbatim}
\usepackage{minitoc}
\usepackage{relsize}

\geometry{
    a4paper,
    % total={210mm,297mm},
    left=20mm,
    right=20mm,
    top=20mm,
    bottom=20mm,
}

\def\imagetop#1{\vtop{\null\hbox{#1}}}

\newcommand\myeq{\stackrel{\mathclap{\normalfont \tiny \mbox{def}}}{=}}
\renewcommand*\thesection{\arabic{section}}
\newcommand{\HRule}{\rule{\linewidth}{0.5mm}} % <--- this defines the vertical lines on the title page

\begin{document}
    \fontfamily{ptm}\selectfont
    \begin{titlepage}
        \begin{center}
            \vspace*{180px}
            \textsc{\Large Very Impressive Text}\\[0.8cm]
            \HRule \\[0.4cm]
            {
                \Huge
                \bfseries
                Collection of important mathematical bullshit and definitions
                \\[0.4cm]
            }
            \HRule \\[1.5cm]
        \end{center}
        \noindent
        \begin{minipage}{0.7\textwidth}
            \begin{flushleft} \large
                \emph{Author:} \\
                Master of Bullshit (M.Bs.) Unknown6656
            \end{flushleft}
        \end{minipage}
    \end{titlepage}
    \newpage
    %---- here begins the table of content ----%
    \setcounter{minitocdepth}{0}
    \dominitoc
    \tableofcontents
    \minitoc
    %---- here ends the table of content ----%
    \newpage
    \setcounter{section}{0}
    \section{VfdA}
    The abbreviation \emph{VfdA} stands for the German expression \emph{Voll für den Ar***} and can be used in pseudo-academic papers and documents like the current one. It is generally used before a long-winded and utterly useless mathematical proof, which stands in no connection to the rest of the paper. It is only used to impress possible readers and to boast about the non-existent knowledge of the author about mathematical subjects. A perfect example for this abbreviation is the following one:
    \\
    \medskip
    \begin{equation*}
        \begin{aligned}
            \textsl{VfdA}: \quad \mathcal{F}(f)(t) = \frac{1}{\left( 2 \pi \right)^\frac{n}{2}} \, \int_{\mathbb{R}^n} \, f(x)e^{-it*x} \, \mathrm{d}x
            \qquad
            \qquad
            \
            \int_{-\infty}^{\infty} \left| f(t) \right| \, \mathrm{d}x < \infty
        \end{aligned}
    \end{equation*}
    \begin{equation*}
        \begin{aligned}
            \begin{matrix*}[l]
                f_m & = & \sum_{k=0}^{n-1} x_{2k} e^{- \frac{2 \pi i}{2 n} m (2k)} + \sum_{k=0}^{n-1} x_{2k + 1} e^{- \frac{2 \pi i}{2 n} m (2k + 1)}
                \\  &   &
                \\  & = & \sum_{k=0}^{n-1} x'_k e^{- \frac{2 \pi i}{2 n} mk} + e^{-\frac{\pi i}{n} m} \, \sum_{k=0}^{n-1} x''_k e^{- \frac{2 \pi i}{n} mk}
                \\  &   &
                \\  & = &
                \begin{cases}
                    f'_m + e^{- \frac{\pi i}{n} m} f''_m             & \text{ if } m < n \\
                    f'_{m-n} - e^{- \frac{\pi i}{n} (m-n)} f''_{m-n} & \text{ if } m \geq n
                \end{cases}
            \end{matrix*}
        \end{aligned}
    \end{equation*}
    \vspace{2mm}
    \\
    One shall note, that \emph{VfdA} automatically implicates \emph{OBdA} (German: \emph{Ohne Beschränkung der Allgemeinheit}, English: \emph{Without loss of generality}) to compensate the so-called \emph{LoC} (English: \emph{Loss of context}) when used.
    \vspace{5mm} \hrule \vspace{5mm}
    \section{Plustorial}
    The \emph{plustorial} (German: \emph{Die Plusultät}) of a number is defined as follows:
    \begin{equation*}
        \begin{aligned}
            n? := \sum_{i = 1}^n i \quad (n \in \mathbb{Z})
            \qquad\qquad
            n? = \sum_{i = 1}^n i = \frac{n (n - 1)}{2}
        \end{aligned}
    \end{equation*}
    \vspace{5mm} \hrule \vspace{5mm}
    \section{Closed Interval}
    The alternate notation for a closed interval over a set $ K \subseteq \mathbb{K} $, which has the comparison operator $ \leq $ defined for every elements $ k,l \in K $, can be written as follows:
    \begin{equation*}
        \begin{aligned}
            \langle k,l \rangle :=
            \begin{cases}
                \left[ k,l \right], & \text{if}\ k \leq l \\
                \left[ l,k \right], & \text{otherwise}
            \end{cases}
            \qquad\qquad
            l,k \in K \subseteq \mathbb{K}
        \end{aligned}
    \end{equation*}
    \begin{equation*}
        \begin{aligned}
            \langle \pm k \rangle := \left[ -k,k \right]
        \end{aligned}
    \end{equation*}
    \vspace{5mm} \hrule \vspace{5mm}
    \section{Set with a finite amount of elements}
    Let $ K $ be the subset of the field $ \mathbb{K} $ and let $ f : K \rightarrow \mathbb{B} $ be a function, which defines for every element $ k \in \mathbb{K} $, whether it is also an element of the subset $ K $.
    \begin{equation*}
        \begin{aligned}
            \forall k \in \mathbb{K} : f (k) \Leftrightarrow k \in K
        \end{aligned}
    \end{equation*}
    Based on the equation above, the subset $ K $ can now be re-defined as follows:
    \begin{equation*}
        \begin{aligned}
            K = \{ k \in \mathbb{K} \ |\ f (k) \} \subset \mathbb{K}
        \end{aligned}
    \end{equation*}
    The following notation can be used to indicate, that the subset $ K \subset \mathbb{K} $ has only a finite amount of elements $ k $:
    \begin{equation*}
        \begin{aligned}
            \{ k \in \mathbb{K} \ |\ f (k) \}_{\substack{ < \\ \infty }} \ :\Leftrightarrow \ | \{ k \in \mathbb{K} \ |\ f (k) \} | < \infty
        \end{aligned}
    \end{equation*}
    \vspace{5mm} \hrule \vspace{5mm}
    \section{Assembler command "ABK"}
    The i386 assembler command \verb"ABK" triggers a quadruple-fault, when loaded into the instruction cache during execution and simultaneously short-circuits the machine's DC voltage regulator with the CPU power inlet, causing the CPU to be grilled with with the given DC voltage (usually 240V in Europe). Have Fun! Example usage:
    \vspace{1mm}
    \fontfamily{pcr}\selectfont
    \definecolor{mygreen}{rgb}{0,0.6,0}
    \definecolor{mymauve}{rgb}{0.58,0,0.82}
    \lstset{
        language={[x86masm]Assembler},
        commentstyle=\color{mygreen},
        otherkeywords={*,abk},
        numberstyle=\tiny\color{mymauve}
    }
    \begin{lstlisting}
    mov   dword ptr [ebp-18h],esp
    push  1
    call  dword ptr ds:[404090h]
    add   esp,4
    mov   dword ptr ds:[403030h],0FFFFFFFFh
    mov   ecx,dword ptr ds:[403020h]
    call  dword ptr ds:[404088h]
    mov   edx,dword ptr ds:[403028h]
    mov   dword ptr [eax],edx
    mov   dword ptr ds:[403038h],ecx
    mov   eax,[403010]
    call  dword ptr ds:[404080h]
    add   esp,4
    call  401C60
    push  403008h
    add   esp,8
    mov   edx,dword ptr ds:[403024h]
    mov   dword ptr [ebp-28h],edx
    push  eax
    mov   ecx,dword ptr ds:[403020h]
    lea   ecx,dword ptr [ebp-10h]
    abk   // initiate quadruple-faulting
    \end{lstlisting}
    \fontfamily{ptm}\selectfont
    \vspace{5mm} \hrule
    \newpage
    \section{$ \varepsilon $-Potato}
    The so-called \emph{Epsilon-Kartoffel} (German expression for \emph{epsilon-potato}) is a special form of an open topological $\varepsilon$-sphere or $\varepsilon$-neighbourhood. It is a subset of the topological space $ \mathbb{K}^n $, which is grouped around the element $ m \in \mathbb{K}^n $. The following rules apply for a subset $ K_{\varepsilon}(m) \subset \mathbb{K}^n $ being qualified as a \emph{epsilon-potato} around the point $ m $:
    \\
    \begin{equation*}
        \begin{aligned}
            \text{(1)} \quad & m \in \mathbb{K}^n \ , \ m \in K_{\varepsilon}(m)
            \\
            \text{(2)} \quad & {K_{\varepsilon}(m)}_{\substack{ < \\ \infty }}
            \\
            \text{(3)} \quad & K_{\varepsilon}(m), \mathfrak{S}(K_{\varepsilon}(m)) \subseteq \mathcal{C}^\infty (\mathbb{K}^n)
            \\
            \text{(4)} \quad & \forall p \in \mathfrak{S}(K_{\varepsilon}(m)) : \nexists q \in K_{\varepsilon}(m) : \exists k \in \mathbb{K} : \vec{mp} * k = \vec{mq} \wedge \|p-m\| \leq \|p-q\|
            \\
            \text{(5)} \quad & \forall p \in K_{\varepsilon}(m) : \exists q \in \mathbb{K}^n \backslash K_{\varepsilon}(m) : \frac{\sup(\|p-m\|)}{\inf(\|q-m\|)} < \infty
            \\
            \text{(6)} \quad & \forall p \in K_{\varepsilon}(m) : \|p-m\| < \infty
        \end{aligned}
    \end{equation*}
    \\
    As the requirements $ (3) $ and $ (4) $ state, the surface $ \mathfrak{S}(K_{\varepsilon}(m)) $ must be an absolute continuously one. It can also be represented by the following function $ \mathcal{S} $:
    \begin{equation*}
        \begin{aligned}
            \mathcal{S} : \mathbb{K}^n \rightarrow \mathfrak{S}(K_{\varepsilon}(m)) \qquad \mathcal{S} \in \mathcal{C}^\infty (\mathbb{K}^n)
        \end{aligned}
    \end{equation*}
    which is an absolute continuous one over the field $ \mathbb{K}^n $ and represents each point on the potato's surface $ \mathfrak{S}(K_{\varepsilon}(m)) $ based on the given $ n $-dimensional rotation angle $ \varphi \in \mathbb{K}^n $.
    \newline
    The point $ m $ is also defined as the \emph{physical center of mass} of the $\varepsilon$-potato $ K_{\varepsilon}(m) $.
    \vspace{5mm} \hrule
    \section{<to be defined>}
    \textsl{<to be defined>}
\end{document}
